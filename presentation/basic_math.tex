%%%%%%%%%%%%%%%%%%%%%%%%%%%%%%%%%%%%%%%%%%%%%%%%%%%%%%%%%%%%%%%%%%%%%%%%%%%%%%%
\subsection{Typesetting Mathematics}
%%%%%%%%%%%%%%%%%%%%%%%%%%%%%%%%%%%%%%%%%%%%%%%%%%%%%%%%%%%%%%%%%%%%%%%%%%%%%%%

%%%%%%%%%%%%%%%%%%%%%%%%%%%%%%%%%%%%%%%%%%%%%%%%%%%%%%%%%%%%%%%%%%%%%%%%%%%%%%%
%%%%%%%%%%%%%%%%%%%%%%%%%%%%%%%%%%%%%%%%%%%%%%%%%%%%%%%%%%%%%%%%%%%%%%%%%%%%%%%
%%%%%%%%%%%%%%%%%%%%%%%%%%%%%%%%%%%%%%%%%%%%%%%%%%%%%%%%%%%%%%%%%%%%%%%%%%%%%%%
\begin{frame}[fragile]{\insertsubsection{}: Dollar Signs}
    \begin{itemize}
        \item Why are dollar signs \keystrokebftt{\$} special? We use them to mark mathematics in text.\\[1ex]
              \begin{figure}
                  \begin{minipage}{0.5\textwidth}
                      \begin{lstlisting}
% not so good:
Let a and b be distinct positive
integers, and let c = a - b + 1.

% much better:
Let $a$ and $b$ be distinct positive
integers, and let $c = a - b + 1$.
                    \end{lstlisting}
                  \end{minipage}
                  \begin{minipage}{0.35\textwidth}
                      % not so good:
                      Let a and b be distinct positive
                      integers, and let c = a - b + 1.

                      % much better:
                      Let $a$ and $b$ be distinct positive
                      integers, and let $c = a - b + 1$.
                  \end{minipage}
              \end{figure}
        % \pause
        \item Always use dollar signs in pairs --- one to begin the mathematics, and one
              to end it.
        % \pause
        \item \LaTeX{} handles spacing automatically; it ignores your spaces.
              \begin{table}
                  \begin{tabular}{cl}
                      \verb|Let $y=mx+b$ be \ldots|      & Let $y=mx+b$ be \ldots      \\
                      \verb|Let $y = m x + b$ be \ldots| & Let $y = m x + b$ be \ldots
                  \end{tabular}
              \end{table}
    \end{itemize}
\end{frame}

%%%%%%%%%%%%%%%%%%%%%%%%%%%%%%%%%%%%%%%%%%%%%%%%%%%%%%%%%%%%%%%%%%%%%%%%%%%%%%%
%%%%%%%%%%%%%%%%%%%%%%%%%%%%%%%%%%%%%%%%%%%%%%%%%%%%%%%%%%%%%%%%%%%%%%%%%%%%%%%
%%%%%%%%%%%%%%%%%%%%%%%%%%%%%%%%%%%%%%%%%%%%%%%%%%%%%%%%%%%%%%%%%%%%%%%%%%%%%%%
\begin{frame}[fragile]{\insertsubsection{}: Notation}
    \begin{itemize}
        \item Use caret \keystrokebftt{\^} for superscripts and underscore \keystrokebftt{\_} for subscripts.
              \begin{table}
                  % \centering
                  \begin{tabular}{c|l}
                      \verb|$y = c_2 x^2 + c_1 x + c_0$| & $y = c_2 x^2 + c_1 x + c_0$
                  \end{tabular}
              \end{table}

              \vskip 2ex

        \item Use curly braces \keystrokebftt{ \{ } \keystrokebftt{ \} } to group
              superscripts and subscripts.
              \begin{table}
                  % \centering
                  \begin{tabular}{c|l}
                      \verb|$F_n = F_n-1 + F_n-2$|       & $F_n = F_n-1 + F_n-2$       \\
                      \verb|$F_{n} = F_{n-1} + F_{n-2}$| & $F_{n} = F_{n-1} + F_{n-2}$
                  \end{tabular}
              \end{table}

        \item There are commands for Greek letters and common notation.
              \begin{table}
                  % \centering
                  \begin{tabular}{c|l}
                      \verb|$\mu = A e^{Q/RT}$|                 & $\mu = A e^{Q/RT}$                 \\
                      \verb|$\Omega = \sum_{k=1}^{n} \omega_k$| & $\Omega = \sum_{k=1}^{n} \omega_k$
                  \end{tabular}
              \end{table}

    \end{itemize}
\end{frame}

%%%%%%%%%%%%%%%%%%%%%%%%%%%%%%%%%%%%%%%%%%%%%%%%%%%%%%%%%%%%%%%%%%%%%%%%%%%%%%%
%%%%%%%%%%%%%%%%%%%%%%%%%%%%%%%%%%%%%%%%%%%%%%%%%%%%%%%%%%%%%%%%%%%%%%%%%%%%%%%
%%%%%%%%%%%%%%%%%%%%%%%%%%%%%%%%%%%%%%%%%%%%%%%%%%%%%%%%%%%%%%%%%%%%%%%%%%%%%%%
\begin{frame}[fragile]{\insertsubsection{}: Displayed Equations}
    \begin{itemize}
        \item If it's big and scary, \emph{display} it on its own line using
              \cmdbegin{equation} and \cmdend{equation}.\\[2ex]
              \begin{figure}
                  \begin{minipage}{0.45\textwidth}
                      \begin{lstlisting}
The roots of a quadratic 
equation are given by
\begin{equation}
    x = \frac{
        -b \pm \sqrt{b^2 - 4ac}
        }{2a}
\end{equation}
where $a$, $b$ and $c$ are \ldots
            \end{lstlisting}
                  \end{minipage}
                  \begin{minipage}{0.4\textwidth}
                      The roots of a quadratic equation
                      are given by
                      \begin{equation}
                          x = \frac{-b \pm \sqrt{b^2 - 4ac}}
                          {2a}
                      \end{equation}
                      where $a$, $b$ and $c$ are \ldots
                  \end{minipage}
              \end{figure}
              \vskip 1em
                  {\scriptsize Caution: \LaTeX{} mostly ignores your spaces in mathematics, but it
                      can't handle blank lines in equations --- don't put blank lines in your
                      mathematics.}
    \end{itemize}
\end{frame}


%%%%%%%%%%%%%%%%%%%%%%%%%%%%%%%%%%%%%%%%%%%%%%%%%%%%%%%%%%%%%%%%%%%%%%%%%%%%%%%
\subsection{Extended \LaTeX}
%%%%%%%%%%%%%%%%%%%%%%%%%%%%%%%%%%%%%%%%%%%%%%%%%%%%%%%%%%%%%%%%%%%%%%%%%%%%%%%

%%%%%%%%%%%%%%%%%%%%%%%%%%%%%%%%%%%%%%%%%%%%%%%%%%%%%%%%%%%%%%%%%%%%%%%%%%%%%%%
%%%%%%%%%%%%%%%%%%%%%%%%%%%%%%%%%%%%%%%%%%%%%%%%%%%%%%%%%%%%%%%%%%%%%%%%%%%%%%%
%%%%%%%%%%%%%%%%%%%%%%%%%%%%%%%%%%%%%%%%%%%%%%%%%%%%%%%%%%%%%%%%%%%%%%%%%%%%%%%
\begin{frame}[fragile]{\insertsubsection{}: Environments}
    \begin{itemize}
        \item \bftt{equation} is an \emph{environment} --- a context.
        \item A command can produce different output in different contexts.

              \begin{figure}
                  \begin{minipage}{0.5\textwidth}
                      \begin{lstlisting}
We can write
$ \Omega = \sum_{k=1}^{n} \omega_k $
in text, or we can write
\begin{equation}
    \Omega = \sum_{k=1}^{n} \omega_k
\end{equation}
to display it.
                \end{lstlisting}
                  \end{minipage}
                  %   \quad
                  \begin{minipage}{0.4\textwidth}
                      We can write
                      $ \Omega = \sum_{k=1}^{n} \omega_k $
                      in text, or we can write
                      \begin{equation}
                          \Omega = \sum_{k=1}^{n} \omega_k
                      \end{equation}
                      to display it.
                  \end{minipage}
              \end{figure}
              \vskip 2ex
        \item Note how the $\Sigma$ is bigger in the \bftt{equation} environment, and how the subscripts and superscripts change position, even though we used the same commands.

                  %   \vskip 1em
                  {\scriptsize In fact, we could have written \bftt{\$...\$} as
                      \cmdbegin{math}\bftt{...}\cmdend{math}.}
    \end{itemize}
\end{frame}

%%%%%%%%%%%%%%%%%%%%%%%%%%%%%%%%%%%%%%%%%%%%%%%%%%%%%%%%%%%%%%%%%%%%%%%%%%%%%%%
%%%%%%%%%%%%%%%%%%%%%%%%%%%%%%%%%%%%%%%%%%%%%%%%%%%%%%%%%%%%%%%%%%%%%%%%%%%%%%%
%%%%%%%%%%%%%%%%%%%%%%%%%%%%%%%%%%%%%%%%%%%%%%%%%%%%%%%%%%%%%%%%%%%%%%%%%%%%%%%
\begin{frame}[fragile]{\insertsubsection{}: Environments}
    \begin{itemize}
        \item The \cmdbs{begin} and \cmdbs{end} commands are used to create many
              different environments.
              \vskip 2ex

        \item The \bftt{itemize} and \bftt{enumerate} environments generate lists.

              \begin{figure}
                  \begin{minipage}{0.5\textwidth}
                      \begin{lstlisting}
\begin{itemize} % for bullet points
\item Biscuits
\item Tea
\end{itemize}

\begin{enumerate} % for numbers
\item Biscuits
\item Tea
\end{enumerate}
            \end{lstlisting}
                  \end{minipage}
                  \begin{minipage}{0.4\textwidth}
                      \begin{itemize} % for bullet points
                          \item Biscuits
                          \item Tea
                      \end{itemize}

                      \begin{enumerate} % for numbers
                          \item Biscuits
                          \item Tea
                      \end{enumerate}
                  \end{minipage}
              \end{figure}
    \end{itemize}
\end{frame}

%%%%%%%%%%%%%%%%%%%%%%%%%%%%%%%%%%%%%%%%%%%%%%%%%%%%%%%%%%%%%%%%%%%%%%%%%%%%%%%
%%%%%%%%%%%%%%%%%%%%%%%%%%%%%%%%%%%%%%%%%%%%%%%%%%%%%%%%%%%%%%%%%%%%%%%%%%%%%%%
%%%%%%%%%%%%%%%%%%%%%%%%%%%%%%%%%%%%%%%%%%%%%%%%%%%%%%%%%%%%%%%%%%%%%%%%%%%%%%%
\begin{frame}[fragile]{\insertsubsection{}: Packages}

    \begin{itemize}
        \item All of the commands and environments we've used so far are built into \LaTeX.
        \item \emph{Packages} are libraries of extra commands and environments. There
              are thousands of freely available packages.
        \item We have to load each of the packages we want to use with a
              \cmdbs{usepackage} command in the \emph{preamble}.
        \item Example: \bftt{amsmath} from the American Mathematical Society.
              \begin{lstlisting}
\documentclass{article}
\usepackage{amsmath} % preamble
\begin{document}
% now we can use commands from amsmath here...
\end{document}
            \end{lstlisting}
    \end{itemize}
\end{frame}

%%%%%%%%%%%%%%%%%%%%%%%%%%%%%%%%%%%%%%%%%%%%%%%%%%%%%%%%%%%%%%%%%%%%%%%%%%%%%%%
\subsection{amsmath Package}
%%%%%%%%%%%%%%%%%%%%%%%%%%%%%%%%%%%%%%%%%%%%%%%%%%%%%%%%%%%%%%%%%%%%%%%%%%%%%%%

%%%%%%%%%%%%%%%%%%%%%%%%%%%%%%%%%%%%%%%%%%%%%%%%%%%%%%%%%%%%%%%%%%%%%%%%%%%%%%%
%%%%%%%%%%%%%%%%%%%%%%%%%%%%%%%%%%%%%%%%%%%%%%%%%%%%%%%%%%%%%%%%%%%%%%%%%%%%%%%
%%%%%%%%%%%%%%%%%%%%%%%%%%%%%%%%%%%%%%%%%%%%%%%%%%%%%%%%%%%%%%%%%%%%%%%%%%%%%%%
\begin{frame}[fragile, allowframebreaks]{\insertsubsection{}: Examples}
    \begin{itemize}
        \item Use \bftt{equation*} (``equation-star'') for unnumbered equations.
              \begin{figure}
                  \begin{minipage}{0.5\textwidth}
                      \begin{lstlisting}
\begin{equation*}
    \Omega = \sum_{k=1}^{n} \omega_k
\end{equation*}
        \end{lstlisting}
                  \end{minipage}
                  \begin{minipage}{0.4\textwidth}
                      \begin{equation*}
                          \Omega = \sum_{k=1}^{n} \omega_k
                      \end{equation*}
                  \end{minipage}
              \end{figure}
        \item \LaTeX{} treats adjacent letters as variables multiplied together, which
              is not always what you want. \bftt{amsmath} defines commands for many common
              mathematical operators.
              \begin{figure}
                  \begin{minipage}{0.5\textwidth}
                      \begin{lstlisting}[basicstyle=\scriptsize]
\begin{equation*} % bad!
    min_{x,y} (1-x)^2 + 100(y-x^2)^2
\end{equation*}
\begin{equation*} % good!
    \min_{x,y}{(1-x)^2 + 100(y-x^2)^2}
\end{equation*}
        \end{lstlisting}
                  \end{minipage}
                  \begin{minipage}{0.4\textwidth}
                      \begin{equation*} % bad!
                          min_{x,y} (1-x)^2 + 100(y-x^2)^2
                      \end{equation*}
                      \begin{equation*} % good!
                          \min_{x,y}{(1-x)^2 + 100(y-x^2)^2}
                      \end{equation*}
                  \end{minipage}
              \end{figure}
\vspace{1cm}

{\small 
        \item You can use \cmdbs{operatorname} for others.
              \begin{figure}
                  \begin{minipage}{0.35\textwidth}
                      \begin{lstlisting}[basicstyle=\tiny]
\begin{equation*}
\beta_i =
\frac{\operatorname{Cov}(R_i, R_m)}
        {\operatorname{Var}(R_m)}
\end{equation*}
        \end{lstlisting}
                  \end{minipage}
                  \begin{minipage}{0.4\textwidth}
                      \begin{equation*}
                          \beta_i =
                          \frac{\operatorname{Cov}(R_i, R_m)}
                          {\operatorname{Var}(R_m)}
                      \end{equation*}
                  \end{minipage}
              \end{figure}

        \item Align a sequence of equations at the equals sign
            \begin{figure}
                \begin{minipage}{0.4\textwidth}
                    \begin{lstlisting}[basicstyle=\scriptsize]
\begin{align*}
(x+1)^3 &= (x+1)(x+1)(x+1) \\
        &= (x+1)(x^2 + 2x + 1) \\
        &= x^3 + 3x^2 + 3x + 1
\end{align*}
        \end{lstlisting}
                \end{minipage}
                \begin{minipage}{0.4\textwidth}
                    \begin{align*}
                        (x+1)^3 & = (x+1)(x+1)(x+1)     \\
                                & = (x+1)(x^2 + 2x + 1) \\
                                & = x^3 + 3x^2 + 3x + 1
                    \end{align*}
                \end{minipage}
            \end{figure}
        with the \bftt{align*} environment.

        \item An ampersand \keystrokebftt{\&} separates the left column (before the $=$) from the right column (after the $=$).
        \item A double backslash \keystrokebftt{\bs}\keystrokebftt{\bs} starts a new line.
              
        }
    \end{itemize}
\end{frame}